\documentclass[12pt]{report}
% Set page margins
\usepackage[margin=4cm]{geometry}

\usepackage[]{graphicx}
\usepackage{setspace}
\usepackage{amsmath}
\usepackage{amsthm} % theorems, examples, definitions
\usepackage{commath} % norm
\usepackage{amssymb}
\usepackage{dirtree}
\usepackage{nicematrix}
\singlespace % interlinea singola

\usepackage{hyperref}
\hypersetup{
	colorlinks=true,
	linkcolor=blue,
	filecolor=magenta,
	urlcolor=blue,
}
 
% All page numbers positioned at the bottom of the page
\usepackage{fancyhdr}
\fancyhf{} % clear all header and footers
\fancyfoot[C]{\thepage}
\renewcommand{\headrulewidth}{0pt} % remove the header rule
\pagestyle{fancy}

% Changes the style of chapter headings
\usepackage{titlesec}

\titleformat{\chapter}
   {\normalfont\LARGE\bfseries}{\thechapter.}{1em}{}

% Change distance between chapter header and text
\titlespacing{\chapter}{0pt}{35pt}{\baselineskip}
\usepackage{titlesec}
\titleformat{\section}
	[hang] % \textlessshape\textgreater
	{\normalfont\bfseries\Large} % \textlessformat\textgreater
	{} % \textlesslabel\textgreater
	{0pt} % \textlesssep\textgreater
	{} % \textlessbefore code\textgreater
\renewcommand{\thesection}{} % Remove section references...
\renewcommand{\thesection}{\arabic{section}} %... from sections
\usepackage{titlesec}

\setcounter{tocdepth}{5}
\setcounter{secnumdepth}{5}

% Prevents LaTeX from filling out a page to the bottom
\raggedbottom

\usepackage{tabularx}
\usepackage{booktabs}
\usepackage{color}
\usepackage{xcolor}
\usepackage{enumitem}
\usepackage{amsmath}
\usepackage{subcaption}
\usepackage{physics}
\usepackage{minted}

\theoremstyle{definition}
\newtheorem{definition}{Definition}[section]
\theoremstyle{definition}
\newtheorem{example}{Example}[section]
\newtheorem{theorem}{Theorem}[section]
\newtheorem{corollary}{Corollary}[theorem]
\newtheorem{lemma}[theorem]{Lemma}
\newtheorem*{remark}{Remark}
\newcommand{\iu}{\mathrm{i}\mkern1mu}

\newcommand\scalemath[2]{\scalebox{#1}{\mbox{\ensuremath{\displaystyle #2}}}}

\makeatletter
\@ifpackageloaded{hyperref}%
  {\newcommand{\mylabel}[2]% #1=name, #2 = contents
	{\protected@write\@auxout{}{\string\newlabel{#1}{{#2}{\thepage}%
	  {\@currentlabelname}{\@currentHref}{}}}}}%
  {\newcommand{\mylabel}[2]% #1=name, #2 = contents
	{\protected@write\@auxout{}{\string\newlabel{#1}{{#2}{\thepage}}}}}
\makeatother

\makeatletter
\let\original@algocf@latexcaption\algocf@latexcaption
\long\def\algocf@latexcaption#1[#2]{%
  \@ifundefined{NR@gettitle}{%
	\def\@currentlabelname{#2}%
  }{%
	\NR@gettitle{#2}%
  }%
  \original@algocf@latexcaption{#1}[{#2}]%
}
\makeatother

\newcounter{cases}
\newcounter{subcases}[cases]
\newenvironment{cs}
{
	\setcounter{cases}{0}
	\setcounter{subcases}{0}
	\newcommand{\case}
	{
		\par\indent\stepcounter{cases}\textbf{Case \thecases.}
	}
	\newcommand{\subcase}
	{
		\par\indent\stepcounter{subcases}\textit{Subcase (\thesubcases):}
	}
}
{
	\par
}
\renewcommand*\thecases{\arabic{cases}}
\renewcommand*\thesubcases{\roman{subcases}}

\begin{document}
\begin{titlepage}
	\clearpage\thispagestyle{empty}
	\centering
	\vspace{1cm}

	\includegraphics[scale=0.58]{../../images/unipi-marchio.eps}

	{\normalsize \noindent Dipartimento di Informatica \\
			Corso di Laurea in Informatica \par}
	
	\vspace{2cm}
	{\huge \textbf{Assignment 03} \par }
	\vspace{1cm}
	{\large Competitive Programming and Contests}

	\vspace{3cm}

	\begin{minipage}[t]{0.47\textwidth}
		{\large{Prof. Rossano Venturini}}
	\end{minipage}\hfill\begin{minipage}[t]{0.47\textwidth}\raggedleft
		{\large {Giacomo Trapani - 600124}}
	\end{minipage}

	\vspace{3cm}

	{\normalsize Academic Year 2024/2025 \par}

	\pagebreak
\end{titlepage}
\paragraph*{Exercise 1}
This exercise requires the implementation of an algorithm to find out the maximum
number of attractions which can be visited throughout the holidays given
constraints on the duration of such holidays, the number of attractions
per day and the time complexity of the algorithm which is expected to be
\(O(n \times D^2)\) with \(n\) the number of cities and \(D\) the number of days.

The solution given makes use of dynamic programming:
\begin{itemize}
	\item \textbf{base case} \(n = 1\): we can visit either \(D\) attractions or the
	number of attractions available in the city, whichever is the lowest.
	\item \textbf{n cities} given the solution for the previous \(n-1\) cities:
	for every \(x \in [0; D]\), we can spend \(D-x\) days in the optimal solution
	thus far and \(x\) days in the \(n\)-th city. When we are done, we compare
	against the previous optimal solution.
\end{itemize}

The dynamic programming table has size \(n \times D\).

\paragraph*{Exercise 2}
This exercise requires the implementation of an algorithm to find out the maximum
number of topics which can be explained throughout a course given constraints on
their order in \(O(n\times \log(n))\) with \(n\) the number of courses.

Considering every topic as a pair (beauty, difficulty), we sort the topics for
their beaut, then we run a ``longest increasing sequence" algorithm using their
difficulty as key.

\paragraph*{How to run}
The code can be run via \texttt{cargo run}. The following directory structure is
expected
\dirtree{%
.1 src.
.2 lib.rs.
.2 main.rs.
.1 tests-1.
.1 tests-2.
.1 Cargo.toml.
.1 Cargo.lock.
}
\end{document}