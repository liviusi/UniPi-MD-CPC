\documentclass[12pt]{report}
% Set page margins
\usepackage[margin=4cm]{geometry}

\usepackage[]{graphicx}
\usepackage{setspace}
\usepackage{amsmath}
\usepackage{amsthm} % theorems, examples, definitions
\usepackage{commath} % norm
\usepackage{amssymb}
\usepackage{dirtree}
\usepackage{nicematrix}
\singlespace % interlinea singola

\usepackage{hyperref}
\hypersetup{
	colorlinks=true,
	linkcolor=blue,
	filecolor=magenta,
	urlcolor=blue,
}
 
% All page numbers positioned at the bottom of the page
\usepackage{fancyhdr}
\fancyhf{} % clear all header and footers
\fancyfoot[C]{\thepage}
\renewcommand{\headrulewidth}{0pt} % remove the header rule
\pagestyle{fancy}

% Changes the style of chapter headings
\usepackage{titlesec}

\titleformat{\chapter}
   {\normalfont\LARGE\bfseries}{\thechapter.}{1em}{}

% Change distance between chapter header and text
\titlespacing{\chapter}{0pt}{35pt}{\baselineskip}
\usepackage{titlesec}
\titleformat{\section}
	[hang] % \textlessshape\textgreater
	{\normalfont\bfseries\Large} % \textlessformat\textgreater
	{} % \textlesslabel\textgreater
	{0pt} % \textlesssep\textgreater
	{} % \textlessbefore code\textgreater
\renewcommand{\thesection}{} % Remove section references...
\renewcommand{\thesection}{\arabic{section}} %... from sections
\usepackage{titlesec}

\setcounter{tocdepth}{5}
\setcounter{secnumdepth}{5}

% Prevents LaTeX from filling out a page to the bottom
\raggedbottom

\usepackage{tabularx}
\usepackage{booktabs}
\usepackage{color}
\usepackage{xcolor}
\usepackage{enumitem}
\usepackage{amsmath}
\usepackage{subcaption}
\usepackage{physics}
\usepackage{minted}

\theoremstyle{definition}
\newtheorem{definition}{Definition}[section]
\theoremstyle{definition}
\newtheorem{example}{Example}[section]
\newtheorem{theorem}{Theorem}[section]
\newtheorem{corollary}{Corollary}[theorem]
\newtheorem{lemma}[theorem]{Lemma}
\newtheorem*{remark}{Remark}
\newcommand{\iu}{\mathrm{i}\mkern1mu}

\newcommand\scalemath[2]{\scalebox{#1}{\mbox{\ensuremath{\displaystyle #2}}}}

\makeatletter
\@ifpackageloaded{hyperref}%
  {\newcommand{\mylabel}[2]% #1=name, #2 = contents
	{\protected@write\@auxout{}{\string\newlabel{#1}{{#2}{\thepage}%
	  {\@currentlabelname}{\@currentHref}{}}}}}%
  {\newcommand{\mylabel}[2]% #1=name, #2 = contents
	{\protected@write\@auxout{}{\string\newlabel{#1}{{#2}{\thepage}}}}}
\makeatother

\makeatletter
\let\original@algocf@latexcaption\algocf@latexcaption
\long\def\algocf@latexcaption#1[#2]{%
  \@ifundefined{NR@gettitle}{%
	\def\@currentlabelname{#2}%
  }{%
	\NR@gettitle{#2}%
  }%
  \original@algocf@latexcaption{#1}[{#2}]%
}
\makeatother

\newcounter{cases}
\newcounter{subcases}[cases]
\newenvironment{cs}
{
	\setcounter{cases}{0}
	\setcounter{subcases}{0}
	\newcommand{\case}
	{
		\par\indent\stepcounter{cases}\textbf{Case \thecases.}
	}
	\newcommand{\subcase}
	{
		\par\indent\stepcounter{subcases}\textit{Subcase (\thesubcases):}
	}
}
{
	\par
}
\renewcommand*\thecases{\arabic{cases}}
\renewcommand*\thesubcases{\roman{subcases}}

\begin{document}
\begin{titlepage}
	\clearpage\thispagestyle{empty}
	\centering
	\vspace{1cm}

	\includegraphics[scale=0.58]{../../images/unipi-marchio.eps}

	{\normalsize \noindent Dipartimento di Informatica \\
			Corso di Laurea in Informatica \par}
	
	\vspace{2cm}
	{\huge \textbf{Assignment 02} \par }
	\vspace{1cm}
	{\large Competitive Programming and Contests}

	\vspace{3cm}

	\begin{minipage}[t]{0.47\textwidth}
		{\large{Prof. Rossano Venturini}}
	\end{minipage}\hfill\begin{minipage}[t]{0.47\textwidth}\raggedleft
		{\large {Giacomo Trapani - 600124}}
	\end{minipage}

	\vspace{3cm}

	{\normalsize Academic Year 2024/2025 \par}

	\pagebreak
\end{titlepage}
\paragraph*{Exercise 1}
This exercise requires the implementation of a data structure on which - given
an input array \texttt{A[1; n]} - the following kind of queries take
\(O((n+m)\log n)\) time:
\begin{itemize}
	\item \texttt{update(i, j, T)} that replaces every value \texttt{A[k]} with
	\(\min(A[k], T)\) with \(k \in [i; j]\),
	\item \texttt{max(i, j)} that returns the largest value in \texttt{A[i..j]}.
\end{itemize}

The data structure of choice are segment trees with lazy propagation, for which
the implementation given is vector-based.

First, we use the array \texttt{A} to build the segment tree \texttt{T} (ref.
to method \texttt{from\_vector}). The idea behind the algorithms for
both \texttt{update} and \texttt{max} is to iterate through the nodes of \texttt{T};
each node represents an interval which may partially or totally overlap
with \texttt{(i, j)}, or not overlap at all. For each of these cases, the
algorithms work in similar ways, refer to \texttt{rec\_update} and \texttt{rec\_max}.

\paragraph*{Exercise 2}
This exercise requires the implementation of a data structure on which - given
\texttt{n} intervals (which are referred to as ``segments'') - it is possible
to answer \texttt{m} queries of the following kind in \(O((n+m)\log n)\) time:
\begin{itemize}
	\item \texttt{is\_there(i, j, k)} that returns 1 if there exists a position
	\(p \in [i; j]\) s.t. exactly \(k\) segments contain \(p\), 0 otherwise.
\end{itemize}

The data structure of choice are segment trees with lazy propagation, for which
the implementation given is vector-based.

The algorithm implemented works in a way similar to that of the ones in the first
exercise, refer to \texttt{rec\_is\_there}.

\paragraph*{How to run}
The code can be run via \texttt{cargo run}. The following directory structure is
expected
\dirtree{%
.1 src.
.2 lib.rs.
.2 main.rs.
.1 tests-1.
.1 tests-2.
.1 Cargo.toml.
.1 Cargo.lock.
}
\end{document}