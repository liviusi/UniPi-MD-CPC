\documentclass[12pt]{report}
% Set page margins
\usepackage[margin=4cm]{geometry}

\usepackage[]{graphicx}
\usepackage{setspace}
\usepackage{amsmath}
\usepackage{amsthm} % theorems, examples, definitions
\usepackage{commath} % norm
\usepackage{amssymb}
\usepackage{nicematrix}
\singlespace % interlinea singola

\usepackage{hyperref}
\hypersetup{
	colorlinks=true,
	linkcolor=blue,
	filecolor=magenta,
	urlcolor=blue,
}
 
% All page numbers positioned at the bottom of the page
\usepackage{fancyhdr}
\fancyhf{} % clear all header and footers
\fancyfoot[C]{\thepage}
\renewcommand{\headrulewidth}{0pt} % remove the header rule
\pagestyle{fancy}

% Changes the style of chapter headings
\usepackage{titlesec}

\titleformat{\chapter}
   {\normalfont\LARGE\bfseries}{\thechapter.}{1em}{}

% Change distance between chapter header and text
\titlespacing{\chapter}{0pt}{35pt}{\baselineskip}
\usepackage{titlesec}
\titleformat{\section}
	[hang] % \textlessshape\textgreater
	{\normalfont\bfseries\Large} % \textlessformat\textgreater
	{} % \textlesslabel\textgreater
	{0pt} % \textlesssep\textgreater
	{} % \textlessbefore code\textgreater
\renewcommand{\thesection}{} % Remove section references...
\renewcommand{\thesection}{\arabic{section}} %... from sections
\usepackage{titlesec}

\setcounter{tocdepth}{5}
\setcounter{secnumdepth}{5}

% Prevents LaTeX from filling out a page to the bottom
\raggedbottom

\usepackage{tabularx}
\usepackage{booktabs}
\usepackage{color}
\usepackage{xcolor}
\usepackage{enumitem}
\usepackage{amsmath}
\usepackage{subcaption}
\usepackage{physics}
\usepackage{minted}

\theoremstyle{definition}
\newtheorem{definition}{Definition}[section]
\theoremstyle{definition}
\newtheorem{example}{Example}[section]
\newtheorem{theorem}{Theorem}[section]
\newtheorem{corollary}{Corollary}[theorem]
\newtheorem{lemma}[theorem]{Lemma}
\newtheorem*{remark}{Remark}
\newcommand{\iu}{\mathrm{i}\mkern1mu}

\newcommand\scalemath[2]{\scalebox{#1}{\mbox{\ensuremath{\displaystyle #2}}}}

\makeatletter
\@ifpackageloaded{hyperref}%
  {\newcommand{\mylabel}[2]% #1=name, #2 = contents
	{\protected@write\@auxout{}{\string\newlabel{#1}{{#2}{\thepage}%
	  {\@currentlabelname}{\@currentHref}{}}}}}%
  {\newcommand{\mylabel}[2]% #1=name, #2 = contents
	{\protected@write\@auxout{}{\string\newlabel{#1}{{#2}{\thepage}}}}}
\makeatother

\makeatletter
\let\original@algocf@latexcaption\algocf@latexcaption
\long\def\algocf@latexcaption#1[#2]{%
  \@ifundefined{NR@gettitle}{%
	\def\@currentlabelname{#2}%
  }{%
	\NR@gettitle{#2}%
  }%
  \original@algocf@latexcaption{#1}[{#2}]%
}
\makeatother

\newcounter{cases}
\newcounter{subcases}[cases]
\newenvironment{cs}
{
	\setcounter{cases}{0}
	\setcounter{subcases}{0}
	\newcommand{\case}
	{
		\par\indent\stepcounter{cases}\textbf{Case \thecases.}
	}
	\newcommand{\subcase}
	{
		\par\indent\stepcounter{subcases}\textit{Subcase (\thesubcases):}
	}
}
{
	\par
}
\renewcommand*\thecases{\arabic{cases}}
\renewcommand*\thesubcases{\roman{subcases}}

\begin{document}
\begin{titlepage}
	\clearpage\thispagestyle{empty}
	\centering
	\vspace{1cm}

	\includegraphics[scale=0.58]{../../images/unipi-marchio.eps}

	{\normalsize \noindent Dipartimento di Informatica \\
			Corso di Laurea in Informatica \par}
	
	\vspace{2cm}
	{\huge \textbf{Assignment 01} \par }
	\vspace{1cm}
	{\large Competitive Programming and Contests}

	\vspace{3cm}

	\begin{minipage}[t]{0.47\textwidth}
		{\large{Prof. Rossano Venturini}}
	\end{minipage}\hfill\begin{minipage}[t]{0.47\textwidth}\raggedleft
		{\large {Giacomo Trapani - 600124}}
	\end{minipage}

	\vspace{3cm}

	{\normalsize Academic Year 2024/2025 \par}

	\pagebreak
\end{titlepage}
\paragraph*{Exercise 1}
This exercise requires the implementation of a method to check if a given tree
is a \textbf{binary search tree}, which means that for each node its key K must
be greater than or equal to that of its left child and less than that of its
right child, we shall refer to this as the ``BST condition''.

For explanation purposes, we shall call \textbf{M} the \textit{root} of the tree,
\textbf{L} its \textit{left child} and \textbf{R} its \textit{right child}.
The algorithm implemented checks whether M verifies the BST condition, then it
performs an in-order traversal of the tree and performs the check mentioned above
for each node.
\paragraph*{Exercise 2}
\end{document}